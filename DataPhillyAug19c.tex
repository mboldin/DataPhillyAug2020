\documentclass{beamer}

\newcommand{\tabitem}{~~\llap{\textbullet}~~}
%\newcolumntype{L}[1]{>{\raggedright\let\newline\\\arraybackslash}p{#1}}

\mode<presentation> {

\usepackage{setspace}
\usepackage{graphicx} 
\usepackage{wrapfig}
\usepackage{amsmath}
\usepackage{mathtools}
\usepackage{color}

\usepackage{booktabs} % Allows the use of \toprule, \midrule and \bottomrule in tables

% The Beamer class comes with a number of default slide themes
% which change the colors and layouts of slides. Below this is a list
% of all the themes, uncomment each in turn to see what they look like.

%\usetheme{default}
%\usetheme{Madrid}
%\usetheme{Warsaw}

% As well as themes, the Beamer class has a number of color themes
% for any slide theme. Uncomment each of these in turn to see how it
% changes the colors of your current slide theme.

%\usecolortheme{albatross}
%\usecolortheme{beaver}
%\usecolortheme{beetle}
%\usecolortheme{crane}
%\usecolortheme{dolphin}
%\usecolortheme{dove}
%\usecolortheme{fly}
%\usecolortheme{lily}
%\usecolortheme{orchid}
%\usecolortheme{rose}
%\usecolortheme{seagull}
%\usecolortheme{seahorse}
%\usecolortheme{whale}
%\usecolortheme{wolverine}

%\setbeamertemplate{footline} % To remove the footer line in all slides uncomment this line
%\setbeamertemplate{footline}[page number] % To replace the footer line in all slides with a simple slide count uncomment this line

\setbeamertemplate{navigation symbols}{} % To remove the navigation symbols from the bottom of all slides uncomment this line
}

\title[Topic]{
	How Will We Know We Are in Recovery? \\
	{\small 
	Applying Data Science to the Analysis of the Business Cycle \\ (and the Covid-19 Pandemic)}
} % The short title appears at the bottom of every slide, the full title is only on the title page

\author[Michael Boldin]{}% Your name
\institute[Fox School of Business, Temple University] % Your institution as it will appear on the bottom of every slide, may be shorthand to save space
{
Michael Boldin, PhD. \\	
Fox School of Business, Temple University\\
Department of Statistical Science 
}
\date{August 19, 2020 \\ DataPhilly} 

\setbeamerfont{caption}{size=\tiny}

\begin{document}
\setbeamercolor{itemize item}{fg=blue}

\begin{frame}
\titlepage % Title page as the first slide
\end{frame}


\begin{frame}
	
	Outline / todo
	
	--Intro
	
	--Chart 1a, 1b, 1c
	
	--terms
	
    --Charts2
    
    x-- SS equations
    -- DFM results
    
    x--MSM slide
    -- MSM unrate prob

    x--DFM 2 slide
    
    -- MSA versions
    -- service sector employ by msa
     
    	
%	to add
%	Unrate, payroll emp
%	Unemplyment claims
%	Service sector
%	Regional charts   	
%	Headlines February
%	NBER methods
%
%	CI, linear DFM
%	nonlinear DFM
%	basic MSM
%	expanded dynamic switching/clustering 
%	SIR
%	US map, PA-NY-NJ
%	PLM results
%	Data ideas
	
\end{frame}


\begin{frame}
	
	Research Project
	{\Large How Will We Know We Are in Recovery?} 
	
	\medskip
	Underlying views, facts, assumptions
	\begin{enumerate}
	\item Recession started in March.
	\item This recession is different, and current economic data resembles outliers (relative time series history).  
	\item A true economic recovery will not occur until the public feels the end to the pandemic is insight or over.
	\item Need to use the data (Covid infections and economic indicators) we have, which is limited in many ways.
	\end{enumerate}
		
\end{frame}


\begin{frame}
	
\begin{figure}
	\centering
	\includegraphics[width=0.95\linewidth]{images/UNRATE1a}
\end{figure}

\end{frame}

\begin{frame}
	
	\begin{figure}
		\centering
		\includegraphics[width=0.95\linewidth]{images/UNRATE1b}
	\end{figure}
	
\end{frame}

\begin{frame}
	
	\begin{figure}
		\centering
		\includegraphics[width=0.95\linewidth]{images/UNRATE1c}
	\end{figure}
	
\end{frame}

\begin{frame}
	\frametitle{Introduction: How Will We Know We Are in Recovery?}
	
	What I will cover:
	\begin{enumerate}
			\item Advice -- applying data science to a research project.  
			\item Explain how the economics profession looks at business cycles -- not necessarily the same as Wall Street chief economists and other quoted in the business press or on TV.
			\item Discuss  and show available data:  economic indicators and publicly available Covid stats (mainly the Johns Hopkins set).
			\item Present modeling ideas and  preliminary analysis steps, including why analysis at the County and Metropolitan (MSA) level and not the State level) is appropriate.	
			\item Discuss insights from theoretical models.		
	\end{enumerate}

Suggestions, comments, and help are welcome
GITHUB: https://github.com/mboldin/DataPhillyAug2020

\end{frame}

\begin{frame}
	\frametitle{More?}
	
Joint research with Prof. Marc Sobel (Temple Univ.) \\
    -- Dirichlet Process Clustering and Bayesian Perspective  

\medskip
\medskip

If I had more time:
\begin{itemize}
	\item Talk about how seasonal effects\\
	 and seasonal adjustments to\\
	 economic data matter.  
	\item Discuss estimation of statistical clustering\\
	 and the bayesian advantages.
	\item Discuss current research by others on the topic.
	\item Nitty gritty of government data sources \\
	and alternatives.
	\item Rant on how the BLS and Census should join\\
	 the 21st century to produce better \\
	 and more timely labor force statistics.
\end{itemize}
			
\end{frame}

\begin{frame}
	
	\begin{figure}
		\centering
		\includegraphics[width=0.70\linewidth]{images/data-scientist-venn-diagram}
		%\caption[Data Science Venn Diagram]{}
	\end{figure}

{\tiny 
Graphic by Steven Kolassa,\\   
compare to Drew Conway diagram of Math and Statistics, Hacking, Substantive Expertise
}
 
\end{frame}
		
\begin{frame}
\frametitle{Project Planning and Data Science Skills}

\begin{wrapfigure}{r}{0.25\linewidth}
	
		\includegraphics[width=1.0\linewidth]{images/data-scientist-venn-diagram}

\end{wrapfigure}


\medskip

Some Advice

{\small
\begin{itemize}
	\item 
	Bad data management and misunderstanding the data often ruins good projects.
	\item 
	Learn at least two programming languages. EXCEL does not count.
	\item 
	Area expertise and understanding 'theory' matters--especially in understanding data limitations. 
	\item
	To find a modeling sweet spot between overly simple models and overfitting, think hard about the applicable math and statistic methods. 
	\item 
	Data visualization helps at all stages, but don't pretend fancy charts alone yield good communication.   
\end{itemize}
}

\end{frame}


\begin{frame}
	\frametitle{Business Cycle Tracking}

{\large Dating Recessions}

\begin{itemize}
	\item Rule of thumb: GDP 2-Quarter rule (for recessions),\\ Unemployment rate movements, Coincident Index patterns. 
	\item NBER Committee -- decides on peak and trough dates with no formal rules.
	\item Dynamic Factor Model (linear)
	\item Switching Model (nonlinear)
	\item Combined: Factor(s) + Switching
\end{itemize}	

\end{frame}

\begin{frame}
	\frametitle{Dynamic Factor Model}
	
	Stock-Watson  Coincident Index
	
\medskip
State Equation -- Common Unobserved Dynamic Factor	\\

\hspace{.25in} 
$ c_{t} = a * c_{t-1} + v_{t} $  
\hspace{.25in} 
$  v \sim \mathcal{N}(0,1)  $
	
\medskip
Measurement Equations 	\\

\hspace{.25in} 
$ y_{i,t} =  b_{i} * c_{t} + e_{i,t}  $
\hspace{.25in} 
$  e \sim \mathcal{N}(0,\Sigma)  $
\\

\hspace{.25in} 
for $i = 1,2, .. K$ indicators
	
\medskip
Parameters estimated using 
\begin{itemize}
	\item 
	Kalman Filter
	\item 
	Maximum Likelihood Estimation (MLE)
\end{itemize}
	
\medskip
I used the Python statsmodels module \\
and the DynamicFactor() method

\end{frame}


\begin{frame}
	\frametitle{Switching Model}
	
Measurement Equation -- GDP or unemployment rate
\\
\medskip
\hspace{.25in} 
 $ un_{i,t} =  a(s_{t}) + b(s_{t}) * un_{t-1} + e_{t}   $  
\hspace{.25in}  
 $  e \sim \mathcal{N}(0,\sigma^2)  $
 	
\medskip
\hspace{.25in} 
s = 1 Recession, unemployment rate tends to rise \\
\hspace{.25in} 
s = 2 Expansion, unemployment rate tends to fall \\
  
\medskip
Markov process probability of regime switching:
\\
\hspace{.25in} 
    $ p(s_{t+1}=1|s_{t}=2)$,
    \hspace{.10in}
    $ p(s_{t+1}=2|s_{t}=1) $
\\

\textit{expansion-to-recession} or  \textit{recession-to-expansion} 
\\
Only depends on the current 'state' 

\medskip
\medskip
More than 2 states or regimes are possible: \\
\hspace{.25in} 
  3: recession-recovery-expansion 
\\
\hspace{.25in} 
  4: recession-recovery-expansion + stagnation 
\\
       
\end{frame}

\begin{frame}
	\frametitle{Dynamic Factor Model + Regime Switching}
	
\medskip
State Equation -- Common Unobserved Dynamic Factor	\\
	
\hspace{.25in} 	
	$ c_{t} = a_{0}(s) + a_{1}(s) * c_{t-1} + v_{t} $  
	\hspace{.25in} 
	$  v \sim \mathcal{N}(0,\sigma_s^2)  $
	
\medskip
Measurement Equations 	\\
\hspace{.25in} 
	$ y_{i,t} =  b_{0}(s) + b_{1}(s)  * c_{t} + e_{i,t}  $
	\hspace{.25in} 
	$  e \sim \mathcal{N}(0,\Sigma(s))  $
	\\
\hspace{.25in} 
	for $i = 1,2, .. K$ indicators
	and s = 1 recession or 2 expansion,
	
\medskip
Random innovations in the state equation ($v_t$) \\
are subtle (relatively small), while regime effects (through s) can yield large 'breaks'.
	
	\medskip
	Parameters estimation requires alternative Kalman Filter or a Particle Filter
	
\end{frame}



\begin{frame}
	\frametitle{S-I-R Virus Model}
	
{\large	S-I-R is the simplest case}
	
	\hspace{.20in} 3 containers for N individuals in total population
	
	\begin{itemize}
	\item
	\textbf{S}usceptible (not immune)\\
	
	\item
	\textbf{I}nfectious\\
	
	\item
	\textbf{R}ecovered (and immune) \\
	
	\end{itemize}

\medskip
	Other versions add \\
	\begin{itemize}
	\item
	Exposed before Infectious (S-E-I-R) \\
	\item
	Quarantined and Hospitalized containers 
	\item
	Death besides Recovered \\
	\item
	Return to Susceptible from Recovered is possible  
	\end{itemize}
	
\end{frame}

\begin{frame}
	\frametitle{S-I-R Virus Model}
	
%\onehalfspacing
	
{\large Equations for S, I, \& R} \\

Laws of Motion:\\

	$ dS =  a * I * (S/N) $ \\

	$ dI = a * I * (S/N)   - b * I $ \\
	
	$ dR = b * I $ \\
	
{\small
Fixed parameters\\
	a = number of contacts of N per day \\  \hspace{.30in} * probability a contract of an S individual results in an infection \\	
	b = recovery rate  = 1 / (days infectious)\\
}

\medskip
Easiest to solve the continuous time version.\\
Need to discreetize to 'apply' to data,\\
 but data is not in the S, I, R form 
	
\end{frame}

\begin{frame}
	\frametitle{S-I-R Virus Model}
	
	Insights
	\begin{itemize}
		\item Ratio a/b is the key \\
		A high infection rate  \& long infection time is bad.  
		\item Herd immunity effects: I falls to 0 when $(S/N) * (a/b) < 1$. 
		\item Hard to change number infected \\
		-- but can lower the maximum I/N, \\
		which elongates the infection time (flattening the curve)
		\item 
		Can let the a parameter vary as peiopel change their behavior (social distancing) so R(t) = a(t)/b can be estimated as a time-series.  
	\end{itemize}
	
	
\end{frame}

\begin{frame}
	
	\begin{figure}
		\centering
		\includegraphics[width=0.95\linewidth]{images/SIR1}
	\end{figure}
	
\end{frame}

\begin{frame}	
	\begin{figure}
		\centering
		\includegraphics[width=0.95\linewidth]{images/SIR2}
	\end{figure}	
\end{frame}

\begin{frame}	
	\begin{figure}
		\centering
		\includegraphics[width=0.95\linewidth]{images/SIR3}
	\end{figure}	
\end{frame}

\begin{frame}
	\frametitle{Infection Model Fitting at County Level}
	
Dependent variable:\\
\begin{itemize}
	\item Change in confirmed covid cases per 1000 
\end{itemize}
		
\medskip

Explanatory variables:
\begin{itemize}
	\item lag of change in confirmed covid cases per 1000  (+)
	\item Lag of confirmed covid cases (to date) per 1000 (-?)
	\item Population density per square mile of land (+)
	\item Percent of population with HS diploma (-)
	\item Median household income (-)
	\item Unemployment rate average in 2019 (?) \\
	... 
\end{itemize}
	
In a simple linear panel model, almost all of the county level explanatory variables seem statistically significant, but $R^2$ is only around 0.25. 	

\end{frame}

\begin{frame}
	\frametitle{Infection Model Fitting at County Level}

Dependent variable:\\
\begin{itemize}
	\item Change in confirmed covid cases per 1000 
\end{itemize}

\medskip

Allowing for 'regimes' that represent county groupings  or clusters is attractive for a variety of reasons.
\\
\begin{itemize}
	\item Unsure about the proper number of regimes
	\item So a Dirichlet Process (DP) model with infinite potential clusters is being developed.
	\item Using Bayesian priors to solve estimation problems with the unconstrained model. 
\end{itemize}
   	
\end{frame}


\begin{frame}
	\frametitle{Full Model -- Economic Indicators\\
		 + Covid Cases by County }
	
	
{\large
	Possible County level clusters (regimes)\\
}	
	
\medskip

	3 x 3 x 3 x 3 regime clusters\\
	
	\begin{tabular}{|l|l|l|l|}
		\hline
		Open
		& Low cases
		& Rising infection 
		& Economic recovery
		 \\
		\hline
		Partially open 
		& Modest cases
		& Steady infection 
		& Stagnation 
		\\
		\hline
		Closed 
		& High cases
		& Rising infection 
		& Contraction
\\
\hline
	\end{tabular}
	
	\medskip
	\medskip
	
	The degrees of Open, Infection cases, \\  and Economic contraction or recovery \\
	will depend on what works in making the 'best' cluster sets.  
	
\end{frame}

	
\begin{frame}
		\frametitle{Conclusion}
		
		\begin{itemize}
			\item first item 
			\item second bullet point
			\item last item
		\end{itemize}
\end{frame}


\end{document} 